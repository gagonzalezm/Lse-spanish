% !TEX root = ../pdf/lsj.tex
% [There are multiple lsj.tex files, but the one in ../pdf is the usual one]


\clearpage
\newpage
\begin{center}
{\bf Overview}
\end{center}

\noindent
{\it learning statistics with jamovi} cubre el contenido de una clase introductoria de estadística, como se enseña típicamente a los estudiantes de psicología. El libro discute cómo empezar en jamovi, así como una introducción a la manipulación de datos. Desde una perspectiva estadística, el libro discute primero la estadística descriptiva y los gráficos, seguido de capítulos sobre la teoría de la probabilidad, el muestreo y la estimación, y la prueba de hipótesis nulas. Después de introducir la teoría, el libro cubre el análisis de tablas de contingencia, correlación, pruebas $t$, regresión, ANOVA y análisis factorial. Las estadísticas bayesianas se tratan al final del libro.

Notas del traductor

-Este documento es una traducción del libro de Navarro DJ and Foxcroft DR (2019). learning statistics with jamovi: a tutorial for psychology students and other beginners. (Version 0.70). DOI: 10.24384/hgc3-7p15 [Available from url: http://learnstatswithjamovi.com]

-La traducción no es literal,  el traductor (quien les escribe) es latino-americano (Chileno) por lo cual algunas palabras podrían ser más cercana a ese tipo de español.

-Dado que en el español es más dificil usar el genero neutro y/o lenguaje inclusivo tanto en pronombres, sustantivos y adjetivos, se opta por ir intercalando el femenino y masculino cuando se no sea posible hacer referencia de una manera neutral, para intentar que haya una representación similar de ambos generos en el texto sin hacer críptica o dificultosa la lectura. Estoy consciente, tambien, que hay un mundo más alla del binarismo y acepta que la solución dada no logra ser totalmente inclusiva, esperando ir corrigiendo esto en el futuro, en la medida el lenguaje evolucione

-Además, cuando se expliquen conceptos estadísticos que son usados en el programa o menus del programa Jamovi, se los traducirá en un primer momento al español,  pero luego se seguirá haciendo referencia a los conceptos en ingles, de modo que facilite el aprendizaje de los conceptos tal cual aparecen en el programa y asi facilitar su uso como encaminar tambien al manejo de los conceptos en la lectura de papers o pasar al software R, donde dificilmente se traducirán las funciones o paquetes al español

-Los capitulos se irán traduciendo y actualizando de manera no lineal. El criterio serán los capitulos más esenciales para poder iniciarse con jamovi, esto es se priorizara repasar algunos conceptos estadísticos y de diseños de investigación, como métodos descriptivos, comparativos y relacionales bivariados.

-Las notas al pie serán algunas traducciones de las notas originales, y otras del traductor. En el segundo caso será indicado explicitamente.

-Dudas, comentarios o sugerencias pueden escribir a gagonzalez1@uc.cl o escribirme en twitter @ Gabriel00602805. Soy Psicólogo y Magister en Epidemiologia, no soy de profesion traductor o inteprete ni tampoco he vivido en paises de habla inglesa, por eso sugerencias en alguna traducción o interpretación serán bien recibidas. Me voy apoyando en traductores online para estar seguro y lo voy constrastando en la medida que mi experiencia como profesor me orienta sobre como podría quedar más claro para los estudiantes. Mi motivación es poder elaborar un material para mis estudiantes, pero tambien para todo el resto de personas de habla-hispana en aprender sobre estadistica en psicología.


\vspace{14cm}
\begin{center}
{\bf Citation}
\end{center}

\noindent
Navarro DJ and Foxcroft DR (2020). Aprendiendo estadísticas con jamovi: un tutorial para estudiantes de psicología y otros principiantes.(González-Medina GA, Traductor) (Version 0.70). ]
